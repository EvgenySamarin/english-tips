\section{Lesson - Time and prepositions}

 \subsection{Time in hours}
 Когда речь идет о времени в часах, конкретных частях суток, или времени до конкретного часа, употребляется предлог \textbf{at}:\\
 at noon, at midday, at night, at midnight, at dawn, at sunset, at 3 a.m., at 3 p.m., at half past 3, at quarter past 3, at 5 to 3 etc.
 
 \subsubsection{Ровное время}
 Просто добавь o’clock к числу.\\
 Вместе с o’clock мы можем использовать только указание времени суток: in the morning (утра), in the afternoon (дня) или in the evening (вечера). 
 
	\begin{quote}
		It`s five o`clock\\
		At ten o`clock
	\end{quote}

Когда часы показывают 30 минут. Употребляется выражение half past
\begin{quote}
	It’s half past seven
\end{quote}

Если на часах меньше 30 минут — то мы говорим past (после такого-то часа), если больше 30 минут — то мы говорим to (до такого-то часа). При этом, само слово minutes (минуты) не называется, но подразумевается.

\begin{quote}
	It’s five past four\\
	It’s five to four
\end{quote}

\subsubsection{Примерное время}
Если не известно точное время, используются предлоги about (около) или almost (почти). 
\begin{quote}
	It’s about five — Сейчас около пяти\\
	It’s almost five — Сейчас почти пять 
\end{quote}

\subsubsection{A.M. и P.M.}
 Эти сокращения часто употребляются в англоязычных странах для обозначения времени суток. В них 24 часа, которые можно разделить на первые 12 (с полуночи до полудня) — это будет a.m. (ante meridiem) и на другие 12 (с полудня до полуночи) — это будет p.m. (post meridiem).
 
 \begin{quote}
 	12 p.m. = Полдень (12 часов дня)\\
 	12 a.m. = Полночь (12 часов ночи)
 \end{quote}
 
 \subsection{Части суток}
 Когда речь идет о частях суток (вечер, утро, полдень). То что неопределенно и может быть "размазано" по времени. Ипользуется предлог In.
  \begin{quote}
 	In the morning\\
 	In the afternoon\\
 	In the evening
 \end{quote}

\subsection{Дни и даты}
Если мы говорим о днях, в том числе о конкретных датах — используем предлог on: 
\begin{quote}
	We will see her on Sunday\\
	My vacation begins on Friday\\
	On the 5th of November
\end{quote}

\subsection{Другое время (месяцы, сезоны, столетия, и т.д.)}

Если говорим о другом времени суток (днем или утром), а также о месяцах, годах и временах года — используем предлог in: 
\begin{quote}
	Cats usually sleep in the afternoon\\
	The nights are long in December \\
	The birds leave in late autumn \\
	This town was founded in 1834
\end{quote}

\subsection{Периоды}
Если мы говорим об определенном периоде во времени — используем разные предлоги в зависимости от ситуации: since(с какого-то)/until(по какое-то время), for (на какой-то период), by, from-to, from-until, during (в течении), (with)in
\begin{quote}
	For a week - На неделю\\
	Mary has been sick since yesterday - больна со вчерашнего дня \\
	within an hour - в течении часа
\end{quote}

\subsection{Полезные фразы}
\begin{enumerate} 
	\item all in good time / всему свое время
	\item time heals all wounds / время все лечит
	\item Not the time / hardly the time — не время
	\item There’s no time like the present — сейчас самое подходящее время
	\item To have all the time in the world — иметь много времени
	\item To have no / little time to spare — нет / мало свободного времени
	\item With time to spare — раньше, чем ожидалось
	\item Have time on your hands — иметь много свободного времени (не знать, чем себя занять)
	\item Half the time — почти всегда (в негативном ключе)
	\item To take one’s time — не спешить
	\item (Right / bang / dead) on time — вовремя
	\item Ahead of time — раньше, чем было запланировано
	\item Behind time — позже, чем было запланировано
	\item In no time / in next to no time — очень скоро, быстро
	\item To make good time — быстро добраться куда-либо
	\item To race / work / battle against time — стараться уложиться в срок, хоть у вас и очень мало времени для этого 
	
\end{enumerate}


 