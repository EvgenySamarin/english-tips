\section{Lesson - PRESENT SIMPLE}

 \subsection{Present simple: описание}
 Применяется для констатации фактов или общеизвестных истин, когда говорят о привычках, расписаниях событий. Понять, что используется PS  можно по словам "маркерам", которые часто используются в предложениях:\\
 always, never, every day(week, month, year), seldom(редко), usualy, regularly, rarly, sometimes
 


 \subsubsection{Without action}
 Используется для описаний. Различные формы глагола - Be: am, is, are;
	\begin{quote}
		I am a doctor\\
		He is a doctor
	\end{quote}
\subsubsection{With action} 
 Используется для утверждений. Для 3-его лица (He, she, it) в конец глагола добавляется окончание s
 \begin{quote}
 	I(we, you, they) want be a doctor / Я хочу быть доктором\\
 	He(she, it) wants be a doctor	
 \end{quote}
 
 \subsubsection{Question} 
 With  help verbs do, does (при этом глаголы do не переводятся). Для первого лица - do. Для 3-его лица - does, при этом в для глагола do окончание s в конце глагола не применяется.
  \begin{quote}
 Do I(we, you, they) want be a doctor?\\
 Does she want be a doctor? (без s на конце глагола)
 \end{quote}

 \subsubsection{Negative} 
 With help 'do/does not'/'don`t, doesn`t'. Do not для первого лица. Does not  для 3-его лица, при этом в для глагола do окончание s в конце глагола не применяется.
   \begin{quote}
 	I do not want be a doctor\\
 	He does not want be a doctor
 \end{quote}

  \subsubsection{Negative Question} 
     \begin{quote}
  Do not you want be a doctor?
  Does not he want be a doctor?
  \end{quote}
 
 \subsection{Домашняя работа}
Составить предложения использующие следующий набор слов:
\begin{enumerate} 
	\item A new car / want:
	\begin{quote} 
	 	I buy a new car, when i want, because i can afford(позволить) it to myself 
	\end{quote} 
	\item two children, a dog / have:
	\begin{quote} 
		Two children don`t catch a dog, even though(даже если) they have a sausage slice.
	\end{quote}
	\item exercise, yoga / do:
	\begin{quote} 
		I always do some yoga exercise, when i get home.
	\end{quote}
	\item to the cinema, to the gym / go:
	\begin{quote} 
		Do you go with me to the cinema or to the gym?
	\end{quote}
	\item tea or coffe / drink:
	\begin{quote} 
		She doesn`t decide (решать), what she's drinking a tea or coffee, but I can choose.
	\end{quote}
	\item in a flat, in the city centre / live:
	\begin{quote} 
		Fred lives in the St. Petersburg city in a flat(квартира) with plastic windows.
	\end{quote}
	\item the newspaper, in bed / read
	\begin{quote} 
		Every day I read the newspaper in a bed because it helps sleep for me.
	\end{quote}
	\item a little German, two languages / speak
	\begin{quote} 	
		A little Germans speak two languages, however(однако), it doesn`t help them be tall.
	\end{quote}
	\item your book to class, an umbrella / take
	\begin{quote} 
		Do you take an umbrella, if I take your book to class?
	\end{quote}
	\item economics, for an exam / study
	\begin{quote} 
		Does he study economics for an exam?
	\end{quote}
	\item in an office / work
	\begin{quote} 
		We don`t work in an office, because we missed all deadlines and got fired.
	\end{quote}
	\item to music, to the radio / listen
	\item sorry, hello / say
	\item glasses, jeans / wear
	\item fast food / eat
	\item housework, homework / do
	\item the guitar / play
	\item TV / watch
	\item animals / like
	\item tennis, chess / play
\end{enumerate}
 